\documentclass{article}
\usepackage[utf8]{inputenc}
\usepackage{amsmath}
\usepackage{amsfonts}
\usepackage{graphicx}
\graphicspath{{./images/}}

\title{Exercício resolvido distribuição de Poisson: Determinação da quantidade de bactérias por litro de leite}
\author{Maria Carolina Boer Copstein}
\date{January 2020}
\begin{document}
\maketitle
\section{Introdução}

\hfill

A distribuição de Poisson descreve eventos aleatórios que ocorrem a uma taxa média definida.Representa a probabilidade de um evento ocorrer um dado número de vezes em um determinado período de tempo,ou espaço,volume ou elemento e é descrito pela fórmula abaixo:

\begin{center}
    $\Pr(x)=\frac{\mu^{x}.e^{-\mu}}{x!}$
\end{center}

\hfill

Sendo:

\hfill

x=Número de ocorrências de um determinado evento durante um período de tempo,em uma área,volume ou elemento.

\hfill

$\mu$=Taxa de ocorrência do evento x.

\hfill

$e$ (aproximadamente) $=2,71828$

\hfill

Aplicação:

\hfill

Supondo que a média da quantidade de bactérias seja de 10 mil  por litro de leite e supondo que o padrão para  o leite não ser considerado contaminado seja de até 15 mil bactérias,qual a probabilidade de  um litro de leite estar contaminado?

\hfill

O problema pode ser descrito por uma distribuição de Poisson onde a função de probabilidade é dada por:
\begin{center}
    $\Pr(x)=\frac{10^{x}.e^{-10}}{x!}$
\end{center}


Substituindo os dados do problema:
\hfill

$Pr[x>15]=1-Pr[x=<15]=1-\sum_{x=0}^{15}\frac{10^{x}.e^{-10}}{x!}=1-0,9513=0,0487$

\hfill 

Com isso a probabilidade de se ter um litro de leite contaminado é de $0,0487$.

\hfill

\textbf{Referências:} \

\hfill

1.Reis, E. A., Reis, I. A. (2016). Introdução aos Modelos Probabilísticos Discretos: Binomial,
Hipergeométrico, Binomial Negativo, Geométrico e Poisson. Relatório Técnico do
Departamento de Estatística da Universidade Federal de Minas Gerais. 
\end{document}

