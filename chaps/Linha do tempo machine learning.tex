\documentclass{article}
\usepackage[utf8]{inputenc}
\usepackage{amsmath}
\usepackage{amsfonts}
\usepackage{graphicx}
\usepackage{amssymb}
\graphicspath{{./images/}}

\title{História Machine Learning}
\author{maria.copstein }
\date{January 2020}

\begin{document}

\maketitle
Uma breve linha do tempo sobre a evolução do machine learning:

1943: Warren McCulloch e Walter Pitts criararam um modelo para rede neural baseando-se em matemática e algoritmos.

1958: Frank Rosenblatt cria um algoritmo para o reconhecimento de padrões baseado em uma rede neural computacional de duas camadas usando simples adição e subtração. 

1980: Kunihiko Fukushima propõe  uma rede neural de hierarquia, multicamada, que foi utilizada em problemas de reconhecimento de padrões.

1989: Os cientistas criaram algoritmos que usavam redes neurais profundas, mas os tempos de treinamento eram medidos em dias.

1992: Juyang Weng publica o Cresceptron, um método para realizar o reconhecimento de objetos 3-D automaticamente a partir de cenas desordenadas.

Meados dos anos 2000: o termo “aprendizagem profunda” começa a ganhar popularidade após um artigo de Geoffrey Hinton e Ruslan Salakhutdinov mostrar como uma rede neural de várias camadas poderia ser pré-treinada uma camada por vez.

2009: acontece o NIPS Workshop sobre Aprendizagem Profunda para Reconhecimento de Voz e descobre-se que com um conjunto de dados suficientemente grande, as redes neurais não precisam de pré-treinamento e as taxas de erro caiam significativamente.

2012: Algoritmos de reconhecimento de padrões artificiais alcançam desempenho em nível humano em determinadas tarefas. E o algoritmo de aprendizagem profunda do Google é capaz de identificar gatos.

2015: Facebook coloca a tecnologia de aprendizado profundo – chamada DeepFace – em operação para marcar e identificar automaticamente usuários do Facebook em fotografias. Algoritmos executam tarefas superiores de reconhecimento facial usando redes profundas que levam em conta 120 milhões de parâmetros.

2016: O algoritmo do Google DeepMind, AlphaGo, mapeia a arte do complexo jogo de tabuleiro Go.

2017: Adoção em massa do Deep Learning em diversas aplicações corporativas e mobile, além do avanço em pesquisas. Todos os eventos de tecnologia ligados a Data Science, IA e Big Data, apontam Deep Learning como a principal tecnologia para criação de sistemas inteligentes.

\hfill

\textbf{Referências:} \

\hfill

1.http://deeplearningbook.com.br/uma-breve-historia-das-redes-neurais-artificiais/
\end{document}
