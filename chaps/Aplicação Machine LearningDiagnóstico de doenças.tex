\documentclass{article}
\usepackage[utf8]{inputenc}
\usepackage{amsmath}
\usepackage{amsfonts}
\usepackage{graphicx}
\graphicspath{{./images/}}

\title{Machine Learning no diagnóstico de doenças:Alzheimer}
\author{Maria Carolina Boer Copstein }
\date{January 2020}

\begin{document}
\maketitle

 \hfill
 
 Machine Learning é uma área da inteligência artificial,que se baseia no fato de máquinas poderem ser treinadas a fim de reconhhecer padrões a partir de dados fornecidos e tomar decisões sem necessariamente a ação de um ser humano.
\hfill

Atualmente estão sendo estudados várias aplicações para essa tecnologia entre elas o diagnóstico de doenças.Como por exemmplo a doença de Alzeimer que é uma doença neurodegenerativa que afeta grande parte da população idosa.
\hfill

A partir de imagens de tomografia por emissão de pósitrons uma equipe de pesquisa da Universidade da Califórnia em São Francisco desenvolveu um algoritmo para detectar a doença em até 6 anos antes de aparecerem os sintomas.
\hfill

O sistema se utiliza de um algoritmo de IA que aprende por deep learning(a partir de uma enorme quantidade de dados, a máquina se torna capaz de reconhecer padrões por mais sutis que sejam).Com o avanço da doença ocorre alteração de como os neurônios utilizam a glucose que é perceptível através das imagens de tomografia.Assim a partir da análise desses dados é possível detectar essa alteração e diagnosticar precocemente a doença.
\hfill 

Os resultados do estudo mostraram uma especificidade de $82$ por cento com sensibilidade de 100 por cento e diagnóstico em uma média de 6 anos antes do aparecimento da doença com intervalo de confiança de 95 por cento. Com isso, mostra-se uma estratégia promissora para a área da saúde,porém que ainda neccessita de mais estudos.

\hfill

\textbf{Referências:} \

\hfill

1.INTELIGÊNCIA artificial ajuda a detectar o Alzheimer.
\hfill

[S. l.], 19 fev. 2019. Disponível em:
\hfill

https://www.medis.pt/mais-medis/saude-e-medicina/
\hfill

inteligenciainteligencia-artificial-ajuda-a-diagnostica
\hfill

r-alzheimer/. Acesso em: 30 jan. 2020.

\hfill

2.MACHINE Learning O que é e qual sua importância?. [S.
\hfill

l.], 30 jan. 2019. Disponível em:
\hfill 

https://www.sas.com/pt\_br/insights/analytics/machine-lea
\hfill

rning.html. Acesso em: 30 jan. 2020.


\end{document}
