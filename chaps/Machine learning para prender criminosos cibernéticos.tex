\documentclass{article}
\usepackage[utf8]{inputenc}
\usepackage{amsmath}
\usepackage{amsfonts}
\usepackage{graphicx}
\graphicspath{{./images/}}

\title{Machine learning para prender criminosos cibernéticos}
\author{Maria Carolina Boer Copstein}
\date{January 2020}

\begin{document}
\maketitle

\hfill

Roubos de IP são cada vez mais frequentes, porém os esforços existentes para detectar invasões tendem a analisar  os casos quando já estão em processos.Porém existe a necessidade de se detectar previamente esses ataques rastreando os sequestradores.
\hfill

Pesquisadores do MIT desenvolveram um novo sistema de aprendizagem de máquinas.Ao determinarem algumas qualidades comuns a "sequestradores em série" treinaram máquinas para reconhecer 800 redes suspeitas 
\hfill 

Segundo o texto de referência:"Em um seqüestro de BGP, um agente malicioso convence as redes próximas de que o melhor caminho para alcançar um endereço IP específico é através da rede. Infelizmente, isso não é muito difícil de fazer, pois o próprio BGP não possui procedimentos de segurança para validar que uma mensagem está realmente vindo do local de origem" e "Para identificar melhor os ataques em série, o grupo primeiro extraiu dados de várias listas de mala direta de operadores de rede, além de dados históricos de BGP tirados a cada cinco minutos da tabela de roteamento global. A partir disso, eles observaram qualidades particulares de redes mal-intencionadas e treinaram um modelo de aprendizado de máquina para identificar automaticamente tais comportamentos."
\hfill 

Um dos desafios é o de detectar falsos positivos como por exemplo erro humano.

\hfill

Foram descobertas redes que já atuavam criminalmente a vários anos.E pretende-se com esse projeto por fim aos problemas cibernéticos atuais.
\

\hfill 

\textbf{Referências:} \

\hfill

1.CONNER-SIMONS, Adam. Using machine learning to hunt down cybercriminals. [S. l.], 8 out. 2019. Disponível em: http://news.mit.edu/2019/using-machine-learning-hunt-down-cybercriminals-1009. Acesso em: 30 jan. 2020.

\end{document}
