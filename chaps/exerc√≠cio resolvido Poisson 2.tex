\documentclass{article}
\usepackage[utf8]{inputenc}
\usepackage{amsmath}
\usepackage{amsfonts}
\usepackage{graphicx}
\usepackage{amssymb}
\graphicspath{{./images/}}

\title{Exercício resolvido aproximação de binomial por Poisson}
\author{maria.copstein }
\date{January 2020}

\begin{document}
\maketitle
\hfill 

A distribuição binomial considera uma variável aleatória X que conta o número de sucessos em n ensaios e  a probabilidade p de sucesso em cada evento e é descrita pela função abaixo:
\begin{center}
    $\Pr(X=x)=\frac{n!}{(n-x)!x!}p^x(1-p) ^ {n-x}$
    
\end{center}
A distribuição de Poisson é a probabilidade de ocorrências de sucesso em um determinado intervalo ou região,de um evento que ocorre a uma taxa definida.

\hfill 

a)Em quais condições a distribuição binomial pode ser aproximada pela distribuição de Poisson?

\hfill 

R:Quando o número de ensaios é muito grande $n\rightarrow\infty$ e $p\rightarrow\ 0$ ou seja a probabilidade de sucesso de cada evento é baixa.

\hfill

b)A partir da distribuição binominal deduza a função da distribuição de Poisson.

\begin{center}
   $\displaystyle\lim_{p\rightarrow\ 0  n\rightarrow\infty} \frac{n!}{(n-x)!x!}p^x(1-p) ^ {n-x}\cong \frac{\lambda^{x}.e^{-\lambda}}{x!}$
   sendo ${\lambda}$=np.
   
   
\hfill 
   
\textbf{Referências:} \
   
\hfill

1.http://www.ufjf.br/clecio\_ferreira/files/2013/10/Cap-7-Principais-Variaveis-Aleatorias-Discretas.pdf
    
\end{center}



\end{document}
