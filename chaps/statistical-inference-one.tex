\documentclass{article}
\usepackage[utf8]{inputenc}
\usepackage{amsmath}
\usepackage{amsfonts}

\title{Statistical inference}
\author{Gabriel Aguiar}
\date{September 2019}

\begin{document}

\maketitle

\section{Cox's axioms}

Let $B(x)$ the degree of belief about a logical proposition $x$, $\hat{x}$ the negation of $x$ and $B(x/y)$ the degree of belief about $x$ since $y$ is true. That way:

\begin{itemize}

\item If $B(x) > B(y)$ and $B(y) > B(z)$, then $B(x) > B(z)$.

\item $B(x) = f(B(\hat{x}))$, where $f$ is any function.

\item $B(x,y) = g(B(x/y),B(y))$, where $g$ is any function and $B(x,y)$ is the degree of belief about $x$ and $y$.

\end{itemize}

\section{Cox's theorem}

A function P that satisfies Cox's axioms also satisfies:

\begin{itemize}

\item $P(x) = 1 \Leftrightarrow P(\hat{x}) = 0$

\item $0 \leq P(x) \leq 1$

\item $P(x) = 1 - P(\hat{x})$

\item $P(x,y) = P(x/y) \; P(y)$

\end{itemize}

\section{Probabilistic interpretation of Cox's formulations}

Given the Cox-generated formulations, the probability function fits perfectly as a degree of belief. With this, we can deal with $B(x)$ amid the constructions of Probability theory.

\section{Bayes' theorem}

Taking the probability function P as a degree of belief, we have:

\begin{itemize}

\item $P(H/D) = \frac{1}{\sum\limits_{H_{i}} P(D/H_{i}) \; P(H_{i})} \; P(D/H) \; P(H)$

$H$: Hypothesis; $D$: Data;

$P(H/D)$: Called "later"; $P(D/H)$: Called "likelihood"

$P(H)$: Called "previous"; $P(D/H_{i})$: Called "evidence"

\end{itemize}

\section{Estimation, error bars and confidence intervals}

Take a degree of belief $P(\theta/D,I)$, where $D$ represents the data and $I$ represents the context information. Let $\hat{\theta}$ the estimate for the possible values of $\theta$. That way:

\begin{itemize}

\item $\frac{dP}{d\theta} (\hat{\theta}/D,I) = 0$ and $\frac{d^{2}P}{d\theta^{2}} (\hat{\theta}/D,I) < 0$

That is, $\hat{\theta}$ represents a maximum point of the function $P(\theta/D,I)$.

\item For simplicity of notation, $P(\theta/D,I) = P(\theta/D)$.

\item If we take the natural logarithm of the function $P(\theta/D)$ and expand it into a truncated Taylor polynomial in the second order around $\hat{\theta}$:

$ln \; P(\theta/D) \equiv L(\theta/D) \approx L(\hat{\theta}/D) + \frac{dL}{d\theta} (\hat{\theta}/D) \; (\theta - \hat{\theta}) + \frac{1}{2} \; \frac{d^{2}L}{d\theta^{2}} (\hat{\theta}/D) \; (\theta - \hat{\theta})^{2}$

$\frac{dP}{d\theta} (\hat{\theta}/D) = 0 \Rightarrow \frac{dL}{d\theta} (\hat{\theta}/D) = 0 \Rightarrow P(\theta/D) \approx \frac{1}{Z} \; e^{\frac{1}{2} \; \frac{d^{2}L}{d\theta^{2}} (\hat{\theta}/D) \; (\theta - \hat{\theta})^{2}}$

$Z$ constant

\end{itemize}

\end{document}