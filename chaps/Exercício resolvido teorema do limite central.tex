\documentclass{article}
\usepackage[utf8]{inputenc}
\usepackage{amsmath}
\usepackage{amsfonts}
\usepackage{graphicx}
\usepackage{amssymb}
\graphicspath{{./images/}}

\title{Exercício resolvido teorema do limite central}
\author{Maria Carolina Boer Copstein}
\date{January 2020}

\begin{document}

\maketitle
Pesquisadores afirmam que uma vacina tem eficácia em $80$ por cento dos casos.Com isso 36 indivíduos que tomaram a vacina foram realizar testes. Qual a probabilidade da eficácia ser inferior a $75$ por cento? 

\hfill 

R:Não temos dados sobre a média e desvio padrão dos resultados obtidos pela pesquisa mas sabemos que só existem duas opções para uma pessoa ou está imunizada ou não está e portanto segue uma distribuição binomial. Devido o tamanho amostral pelo teorema do limite central podemos realiar uma aproximação normal,através de uma váriavel Z
\begin{center}
    $ Z=\frac{P-p}{\sqrt{\frac{p(1-p)}{n}})}=\frac{0,75-0,8}{\sqrt{\frac{0,8(0,2)}{36}})}= -0,75$
    
\hfill

    $Pr(P<0,75)=Pr(P< -0,75)=0,2266$
\end{center}
Portanto a probabilidade de menos de $75$ por cento estar imunizado é de $22,66$ por cento.

\hfill 

\textbf{Referências:} \

\hfill 

1.COSTA ALVES, JOSÉ EDUARDO. TEOREMA CENTRAL

\hfill 
DO LIMITE: COMPREENDENDO E APLICANDO. [S. l.], 26 jul. 2016. Disponível em: https://sca.profmat-sbm.org.br/sca\_v2/get\_tcc3.php?id=94792. Acesso em: 30 jan. 2020.





\end{document}
