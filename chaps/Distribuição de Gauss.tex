\documentclass{article}
\usepackage[utf8]{inputenc}
\usepackage{amsmath}
\usepackage{amsfonts}
\usepackage{graphicx}
\usepackage{amssymb}
\graphicspath{{./images/}}


\title{Gaussiana e teorema do limite central}
\author{maria.copstein }
\date{January 2020}

\begin{document}

\maketitle
As distribuições binomiais e de Poisson tendem a uma Gaussiana quando respectivamente n ou ${\mu}$ são grandes.
\hfill

Essa distribuição é dada pela função de probabilidade abaixo:
\begin{center}
    $\Pr(X=x)=\frac{1}{\sigma \surd {2 \pi }}exp\frac{-(x-\mu)^2}{2\sigma^2 }$
    
\end{center}
Teorema do limite central:

\hfill 

Dado um conjunto de médias M\_{n} formado a partir de n amostras de uma população x\_{i} com média finita $\mu$ e variância $\sigma^2$ então a distribuição tende a uma gaussiana com média $\mu$ e desvio padrão $\frac{\sigma}{\surd{n} }$

\hfill

Observações:

\hfill 

1-Fazer médias produz como resultado uma distribuição Gaussiana.

\hfill 

2- A convergência no centro ocorre muito mais rápido que nas laterais.

\hfill 

3-A medida amostral melhora a medida que o tamanho amostral aumenta

\hfill 

Portanto afirma que a soma  de N variáveis aleatórias independentes , com qualquer distribuição e variâncias semelhantes, é uma variável com distribuição que se aproxima da distribuição de Gauss (distribuição normal) quando N aumenta.

\textbf{Referências:} \
   
\hfill

1.http://www.leg.ufpr.br/~silvia/CE001/node38.html

\hfill 

2.https://edisciplinas.usp.br/pluginfile.php/799829/mod
\hfill 

\_resource/content/1/aula2.pdf

\hfill 

3.https://www.if.ufrgs.br/~lang/Textos/Teorema\_Central\_Limite.pdf

    

\end{document}
