\documentclass{article}
\usepackage[utf8]{inputenc}
\usepackage{amsmath}
\usepackage{amsfonts}
\usepackage{graphicx}
\graphicspath{{./images/}}

\title{Exercícios sobre probabilidade: Mega-sena}
\author{Maria Carolina Boer Copstein}
\date{January 2020}

\begin{document}
\maketitle
\section{Introdução}
\hfill
 Jogos de azar são feitos de forma a existir uma maior probabilidade de fracasso do que de sucesso. Sendo assim oferece um prêmio de alto valor com a finalidade de atrair apostadores e um jogo com regras simples. Com base nisso, em cada item apresentado abaixo será calculado a probabilidade de uma pessoa levar o dinheiro em diferentes condições. 

\section{Probabilidade de vencer}
\subsection{Fazendo uma aposta mínima}
Conceito envolvido:
\hfill

Combinação de uma determinada quantidade de números escolhendo uma quantidade menor dentre esses números para formar grupos.

\hfill

Aplicação:
\hfill 

Sabendo que uma cartela de jogo tem 60 números e uma aposta mínima consiste na escolha de 6 números,temos a combinação de 60 escolhendo 6,o que resulta em:

\begin{center}
    $C_{60,6}=\frac{60!}{6!(60-6)!}=50063860$
\end{center}

\hfill 
Com isso existem 50063860 jogos possíveis escolhendo 6 números dentre os 60 possíveis e a probabilidade de a pessoa ganhar fazendo uma aposta mínima pode ser dada por:

\begin{center}
    $\Pr(ganhar)=\frac{1}{50063860}=1.99745*10^{-8}$
\end{center}

Ou seja a probabilidade da pessoa ter o jogo premiado dentre os 50063860 possíveis é de aproximadamente  0.00000002.


\subsection{Sabendo que acertou os três primeiros números}
Conceito envolvido:
\hfill 

Quando um evento que restringe o espaço amostral ocorre, a probabilidade de acontecer um novo evento é denominada probabilidade condicional.

\hfill

Aplicação:
\hfill

Nesse caso sabemos que a pessoa já acertou três números,ou seja,é de se esperar que a probabilidade dela ganhar seja maior 


\begin{center}
    $\Pr(a/b)=\frac{\Pr(a)\cap \Pr(b)}{\Pr(b)}$
\end{center}

\hfill

Aplicando ao caso descrito acima temos:
\begin{center}
    $\Pr(A6/A3)=\frac{\Pr(A6 \cap A3)}{\Pr( A3)}$
    
    
    $\Pr(A6/A3)=\frac{\Pr(6)}{A3}=\frac{\frac{1}{C_{60,6}}}{\frac{C_{6,3}}{C_{60,3}}}=0.000034$
    
    A6=acertar os 6 números
    
    A3=acertar 3 primeiros números
\end{center}

\hfill

Como era de se esperar a probabilidade de ganhar na Mega sena sabendo que acertou os três primeiros números é maior do que simplesmente a de ganhar acertando os 6 números apesar disso o valor ainda é muito baixo.

\subsection{Tempo médio,em anos,para ganhar apostando em 
todos os concursos}
\hfill

Conceitos envolvidos:

\hfill

Esperança matemática:Média ponderada de valores pela sua probabilidade 

\begin{center}
    $E(X)=\frac{1}{p}$
\end{center}

Como são realizados dois concursos semanais são necessárias:

\begin{center}
    $E(X)=\frac{1}{2*\frac{1}{50063860}}=25031930$
\end{center}

Com isso 25031930 semanas jogando é o tempo médio para se  ganhar na mega sena considerando que um ano tem aproximadamente 52 semanas teremos o equivalente de 481.383 anos.

\hfill

\textbf{Referências:} \

\hfill

1.SOUZA PRADO, José William. NOÇÕES DE PROBABILIDADE 
\hfill

POR MEIO DE JOGOS DE AZAR. 2015. UNIVERSIDADE ESTADUAL
\hfill

DE FEIRA DE SANTANA. Disponível em:
\hfill 

https://repositorio.ufpb.br/jspui/bitstream/tede/9474/
\hfill

2/arquivototal.pdf. Acesso em: 29 jan. 2020.

\hfill

2.DEKKING, F.M.; KRAAIKAMP, C.; LOPUHAA, H.P.;MEESTER, L.E.
\hfill 

A Modern Introduction to Probability and Statistics. [S. l.: s. n.], 2005.


\end{document}
